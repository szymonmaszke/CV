\documentclass[../main.tex]{subfiles}

\begin{document}
  \begin{tikzpicture}[remember picture, overlay]
    % Education
    \node (education) at ($(current page.north west)+(\paperwidth-\cvMiddleSidebar+\Margin / 2,\textStart)+(-0.5mm,-1.8\Margin)$)
      [rectangle, fill=education, minimum width=\paperwidth-\cvMiddleSidebar, minimum height=\cvSectionHeight]
      {\Large \textbf{\textcolor{text}{\faGraduationCap\space Education}}};
    \node (education data) at (education.south) [below, text width = \paperwidth-\cvMiddleSidebar-\Margin] {
        \vspace*{-0.35cm}
        \education{M.Sc. Computer Science (Machine Learning specialization)}
        {Jagiellonian University}{\faTrophy\space GPA 4.7/5.0 (ML 4.9+/5.0)}
        {2018}{2021}

        \vspace*{-0.25cm}
        \education{B.Sc. Computer Science}
        {Jagiellonian University}{\faMapMarker\space Faculty of Physics}
        {2014}{2017}
    };

    % Work
    \node (work) at ($(education data.south)+(0cm, -2.3mm)$)
      [rectangle, fill=work, below,
      minimum width=\paperwidth-\cvMiddleSidebar, minimum height=\cvSectionHeight]
      {\Large \textbf{\textcolor{text}{\faBriefcase\space Work}}};
    \node (work data) at (work.south) [below, text width = \paperwidth-\cvMiddleSidebar-\Margin] {
      \begin{cvtext}
        \work{Consultant}{\href{https://moises.ai/}{moises.ai}}{05/2022}{Present}
        {
          {
            Developing conversion of audio neural networks to edge-devices          
          }
        }
        \work{Consultant}{\href{https://www.creativ-ceutical.com/}{Creativ-Ceutical}}{03/2022}{Present}
        {
          {
            Advising on the use of machine learning (ML) and deep learning (DL),
            Laying out projects architecture and development (e.g. workflows)
          }
        }
        \work{Head Of Content}{\href{https://www.theaicore.com/}{AiCore}}{11/2020}{09/2021}
        {
          {
            Developed and taught Linux/ML/DL/DevOps units of the course,
            Ran mock interviews and supported student projects/development
          }
        }
        \work{Freelance Machine Learning}{ArtPlate}{08/2019}{06/2020}
        {
          {Developed cost-effective neural network art tagger (\textbf{see open source})}
        }

        \work{Machine Learning Research}{\href{https://codete.com/}{Codete}}{04/2018}{09/2018}
        {
          {
            Developed \& tested POC Keras \faArrowsH Tensorflow neural network converter,
            Co-created company's commercial Machine Learning \& NLP courses
          }
        }
      \end{cvtext}
    };

    % Open Source
    \node (foss) at ($(work data.south)+(0cm, 2.89mm)$)
      [rectangle, fill=foss, below,
      minimum width=\paperwidth-\cvMiddleSidebar, minimum height=\cvSectionHeight]
      {\Large \textbf{\textcolor{background}{\faCircleONotch\space Open Source}}};
    \node (foss data) at (foss.south) [below, text width = \paperwidth-\cvMiddleSidebar-\Margin] {
      \begin{cvtext}
        \project{szymonmaszke/torchlayers}{https://github.com/szymonmaszke/torchlayers}{\input{data/generated/torchlayers.txt}}{03/2020}
        {Shape \& dimension inference for PyTorch (like \href{https://keras.io/}{\textbf{Keras}})}
        {Improved prototyping speed, zero overhead, featured on \href{https://www.kdnuggets.com/2020/04/pytorch-models-torchlayers.html}{KDNuggets}}

        \fossrulefill

        \project{szymonmaszke/torchlambda}{https://github.com/szymonmaszke/torchlambda}{\input{data/generated/torchlambda.txt}}{03/2020}
        {Lightweight deployment of PyTorch neural networks to AWS Lambda}
        {Reduced fixed costs of AI infrastructure (2M free requests)}

        \fossrulefill

        \project{szymonmaszke/torchdatasets}{https://github.com/szymonmaszke/torchdatasets}{\input{data/generated/torchdata.txt}}{09/2019}
        {Extended PyTorch datasets with cache, map etc. (like \href{https://www.tensorflow.org/guide/data}{\textbf{tensorflow.data}})}
        {\faTrophy\space One of \href{https://devpost.com/software/torch-torchdata-torchfunc}{PyTorch Global Summer Hackathon 2019} winning projects}

      \end{cvtext}

      \vspace*{-0.42cm}

      \clause
    };

  \end{tikzpicture}
\end{document}
